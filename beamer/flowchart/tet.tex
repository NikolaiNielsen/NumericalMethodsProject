\documentclass{article}

\usepackage[latin1]{inputenc}
\usepackage{tikz}
\usetikzlibrary{shapes,arrows}

%%%<
\usepackage{verbatim}
\usepackage[active,tightpage]{preview}
\PreviewEnvironment{tikzpicture}
\setlength\PreviewBorder{5pt}%
%%%>

\begin{comment}
:Title: Simple flow chart
:Tags: Diagrams

With PGF/TikZ you can draw flow charts with relative ease. This flow chart from [1]_
outlines an algorithm for identifying the parameters of an autonomous underwater vehicle model.

Note that relative node
placement has been used to avoid placing nodes explicitly. This feature was
introduced in PGF/TikZ >= 1.09.

.. [1] Bossley, K.; Brown, M. & Harris, C. Neurofuzzy identification of an autonomous underwater vehicle `International Journal of Systems Science`, 1999, 30, 901-913


\end{comment}


\begin{document}
\pagestyle{empty}


% Define block styles
\tikzstyle{decision} = [diamond, draw, fill=blue!20,
    text width=4.5em, text badly centered, node distance=3cm, inner sep=0pt]
\tikzstyle{block} = [rectangle, draw, fill=blue!20,
    text width=5em, text centered, rounded corners, minimum height=4em]
\tikzstyle{line} = [draw, -latex']
\tikzstyle{cloud} = [draw, ellipse,fill=red!20, node distance=3cm,
    minimum height=2em]

\begin{tikzpicture}[node distance = 2cm, auto]
    % Place nodes
    \node [block] (init) {Initialize model};
    \node [cloud, left of=init] (expert) {Input args};
    % \node [cloud, right of=init] (system) {system};
    \node [block, below of=init] (identify) {Set patient zero};
    \node [block, below of=identify] (evaluate) {Find neighbors};
    \node [block, below of=evaluate] (transmit) {Evaluate neighbors, transmit};
    \node [block, left of=evaluate, node distance=3cm] (update) {Update model};
    \node [block, below of=transmit] (cure) {Evaluate sick, kill and cure};
    \node [decision, left of=cure] (decide) {Stop?};
    \node [block, below of=decide, node distance=3cm] (stop) {stop};
    % Draw edges
    \path [line] (init) -- (identify);
    \path [line] (identify) -- (evaluate);
    %\path [line] (evaluate) -- (decide);
    \path [line] (evaluate) -- (transmit);
    \path [line] (transmit) -- (cure);
    \path [line] (decide) -- (update);
    \path [line] (update) |- (evaluate);
    \path [line] (decide) -- (stop);
    \path [line,dashed] (expert) -- (init);
    \path [line] (cure) -- (decide);
    % \path [line,dashed] (system) -- (init);
    % \path [line,dashed] (system) |- (evaluate);
\end{tikzpicture}


\end{document}
