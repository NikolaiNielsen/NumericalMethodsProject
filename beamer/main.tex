% !TEX TS-program = xelatex
% !TEX option = -shell-escape
\documentclass{beamer}
\usepackage{amsmath}
\usepackage{pgf}
\usepackage[english]{babel}
\usepackage{tikz}
\usepackage{pgfplots}
\usepackage{url}
\usepackage[separate-uncertainty=true]{siunitx}
\usepackage{pgfplotsthemetol}
\usepackage{enumitem}
\setbeamertemplate{itemize items}[default]
\usepgfplotslibrary{units}
\usepgfplotslibrary{groupplots}
%\usetikzlibrary{external}
%\tikzexternalize
\pgfplotsset{
  compat=newest,
  unit code/.code 2 args={\si{#1#2}},
  ylabsh/.style={every axis y label/.style={at={(0,0.5)}, xshift=#1, rotate=90}}
}
\usetheme[sectionpage=none]{metropolis}
\title{Herd immunity in a network}
\date{28th of July 2017}
\author{Jens Kinch, Nikolai Plambech Nielsen \& Mads Ehrhorn}
\institute{Niels Bohr Institute \\ University of Copenhagen}
\begin{document}

\maketitle

\section{Introduction}
\frame{
\frametitle{Introduction}
\begin{itemize}
    \item Herd immunity
    \item Disease types
    \item Networks
\end{itemize}



}

\section{Code}
\frame{
\frametitle{Code}
How did we (Nikolai) code the shit outta this motherfucker?

% \begin{figure}
% \begin{tikzpicture}


% Define block styles
\tikzstyle{decision} = [diamond, draw, fill=blue!20,
    text width=4.5em, text badly centered, node distance=3cm, inner sep=0pt]
\tikzstyle{block} = [rectangle, draw, fill=blue!20,
    text width=5em, text centered, rounded corners, minimum height=4em]
\tikzstyle{line} = [draw, -latex']
\tikzstyle{cloud} = [draw, ellipse,fill=red!20, node distance=3cm,
    minimum height=2em]

\begin{tikzpicture}[node distance = 2cm, auto]
    % Place nodes
    \node [block] (init) {Initialize model};
    \node [cloud, left of=init] (expert) {Input args};
    % \node [cloud, right of=init] (system) {system};
    \node [block, below of=init] (identify) {Set patient zero};
    \node [block, below of=identify] (evaluate) {Find neighbors};
    \node [block, below of=evaluate] (transmit) {Evaluate neighbors, transmit};
    \node [block, left of=evaluate, node distance=3cm] (update) {Update model};
    \node [block, below of=transmit] (cure) {Evaluate sick, kill and cure};
    \node [decision, left of=cure] (decide) {Break?};
    \node [block, below of=decide, node distance=3cm] (stop) {Stop};
    % Draw edges
    \path [line] (init) -- (identify);
    \path [line] (identify) -- (evaluate);
    %\path [line] (evaluate) -- (decide);
    \path [line] (evaluate) -- (transmit);
    \path [line] (transmit) -- (cure);
    \path [line] (decide) -- (update);
    \path [line] (update) |- (evaluate);
    \path [line] (decide) -- (stop);
    \path [line,dashed] (expert) -- (init);
    \path [line] (cure) -- (decide);
    % \path [line,dashed] (system) -- (init);
    % \path [line,dashed] (system) |- (evaluate);
\end{tikzpicture}

% \end{figure}

}

\section{Herd immunity}
\frame{
\frametitle{Herd immunity}

\begin{itemize}[label=\textbullet]
    \item \( R_0 \): `basic reproduction number'
    \begin{itemize}
        \item Avg. no. of people infected pr. person
    \end{itemize}
    \item \( p_c \): critical proportion of pop. immunized
    \begin{itemize}
        \item \( p_c = 1 - 1/R_0 \)
    \end{itemize}
    \item Herd immunity passively protects whole population
\end{itemize}

}

\section{Networks}
\frame{
\frametitle{Networks}
The types of networks we be bitchin' up.

I love you. And I miss you.

Walkthrough of random, small world, scale free and custom networks.

}

\section{The simulations}
\frame{
\frametitle{The simulations}
How we simul'ed.

}

\section{Results}
\frame{
\frametitle{Results}

\begin{figure}
    \centering
    \begin{tikzpicture}
        \begin{axis}[
            xlabel=\% immune,
            ylabel=\% of runs with no. of sick \( \leq 5 \),
            title=Ebola, small world, \( N = 1000 \)
            ]
            \addplot [only marks,error bars/.cd,y dir = both,y explicit] table [col sep=comma,x index = {0}, y index = {1}, y error index = {2}] {../nikolai/ebola_smallworld.csv};
        \end{axis}
    \end{tikzpicture}
\end{figure}

}

\end{document}
