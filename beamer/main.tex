% !TEX TS-program = xelatex
% !TEX option = -shell-escape
\documentclass{beamer}
\usepackage{amsmath}
\usepackage{pgf}
\usepackage[english]{babel}
\usepackage{tikz}
\usepackage{pgfplots}
\usepackage{url}
\usepackage[separate-uncertainty=true]{siunitx}
\usepackage{pgfplotsthemetol}

\usepackage{enumitem}
\setitemize{label=\usebeamerfont*{itemize item}%
  \usebeamercolor[fg]{itemize item}
  \usebeamertemplate{itemize item}}

\usepgfplotslibrary{units}
\usepgfplotslibrary{groupplots}
\usetikzlibrary{external}
\tikzexternalize
\pgfplotsset{
  compat=1.13,
  unit code/.code 2 args={\si{#1#2}},
  ylabsh/.style={every axis y label/.style={at={(0,0.5)}, xshift=#1, rotate=90}}
}

\usetheme[sectionpage=none]{metropolis}
\title{Herd immunity in a network}
\date{28th of July 2017}
\author{Jens Kinch, Nikolai Plambech Nielsen \& Mads Ehrhorn}
\institute{Niels Bohr Institute \\ University of Copenhagen}
\begin{document}

\maketitle

\section{Test}
\frame{
\frametitle{Test}
\begin{enumerate}
    \item Forskellige spredningsprocenter
    \item Forskellige vaccinationsprocenter
    \item Forskellige netværkstyper
    \item Herd immunity threshold
    \item Flyrejse?
\end{enumerate}

}

\section{Referencer}
\frame{
\frametitle{Referencer}
\begin{thebibliography}{5}

\footnotesize{}
\bibitem{fakealicious}Fake ref

\end{thebibliography}
}
\end{document}
